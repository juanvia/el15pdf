\documentclass[10pt,a4paper]{article}
\usepackage[utf8]{inputenc}
\usepackage[spanish]{babel}
\usepackage{amsmath}
\usepackage{cases}
\usepackage{graphics}
\usepackage{amsfonts}
\usepackage{amssymb}
\usepackage{comment}
\usepackage[left=2cm,right=2cm,top=2cm,bottom=2cm]{geometry}
\setlength{\parindent}{0pt}
\setlength{\parskip}{5pt}
\begin{document}

\title{Plano perpendicular a la intersección de dos planos}
\author{Juan Vicente Vía Baylet}
\maketitle

\begin{abstract}
  Vamos a ver una manera de resolver un ejercicio en donde se busca el plano perpendicular a la
  intersección entre dos planos y que pase por un punto dado.
  Esta resolución cumple con las condiciones de que no se usen matrices y
  de que no se use el producto vectorial.
\end{abstract}


\section{El problema}


Se consideran los planos $\Pi$ $\rightarrow$  $4x-y+3z=2$ y $\Pi'$ $\rightarrow$ $2x+2y-z=6$.
Hallar la ecuación de un plano que sea ortogonal a $\Pi\cap\Pi'$ y que pase por $(4,2,-1)$.

\section{La solución}

NOTA: En https://www.geogebra.org/3d/a3scf54u se puede ver este problema y su solución en un gráfico 3D interactivo.

\subsection{Bautismos}

Para solucionar un problema es generalmente útil comenzar asegurándose de que las cosas
estén bien identificadas y tengan un nombre. Entonces\dots

\textbf{Vamos a llamar \textbf{$\Lambda^\cap$} a la recta intersección $\Pi\cap\Pi'$.} Nuestra misión es encontrarla para que nos ayude
a llegar al plano buscado.

\textbf{Vamos a llamar $\Pi^\perp$ al plano buscado}, el único entre los infinitos
planos perpendiculares a {$\Lambda^\cap$} que pasa por el punto $(4,2,-1)$.

Tengamos presentes además las ecuaciones que definen los planos originales del problema:

\begin{gather}
  \Pi \rightarrow 4x-y+3z=2 \label{eq:planopi}\\
  \Pi'\rightarrow x+2y-z=6 \label{eq:planopiprima}
\end{gather}

Vamos a referirnos frecuentemente a estas ecuaciones como \textbf{la ecuación (\ref{eq:planopi})} y \textbf{la ecuación (\ref{eq:planopiprima})}


\subsection{Método}


\begin{enumerate}
  \item Primero vamos a encontrar la intersección $\Lambda^\cap$ de los dos planos: $\Pi$ y $\Pi'$:
  \item Luego vamos a encontrar la ecuación de los planos ortogonales a $\Lambda^\cap$ y
        entre ellos seleccionar $\Pi^\perp$, el que pasa por $(4,2,-1)$.
\end{enumerate}


\subsection{La intersección $\Lambda^\cap$}


Como la intersección $\Lambda^\cap$ está en ambos planos $\Pi$ y $\Pi'$ sus puntos deben cumplir simultáneamente
con las ecuaciones que los definen, es decir:

$$
  \Lambda^\cap = (x,y,z) \in \mathbb{R}^3  \text{ para los cuales se cumplen}
  \begin{cases}
    4x-y+3z=2 \text{} \\
    2x+2y-z=6 \text{} \\
  \end{cases}
$$

Eso es una recta. Vamos a expresar la recta en forma paramétrica.
Buscamos entonces $x=f_x(\lambda)$, $y=f_y(\lambda)$ y $z=f_z(\lambda)$ a partir de la anterior restricción que nos permitan calcular
los puntos de esta recta intersección $\Lambda^\cap$ en función de un parámetro $\lambda$.


\subsubsection{Operaciones para poner $x$, $y$ y $z$ en función de $\lambda$. Obtención de la forma paramétrica de la recta de intersección $\Lambda^\cap$.}


Recordemos que partimos de
$$
  \Lambda^\cap = (x,y,z) \in \mathbb{R}^3  \text{ para los cuales se cumplen}
  \begin{cases}
    4x-y+3z=2 \text{} \\
    2x+2y-z=6 \text{}
  \end{cases}
$$
Decidimos tomar $\lambda$ como parámetro y hacemos $\lambda = x$. Obtenemos así directamente la primera de
las ecuaciones paramétricas de $\Lambda^\cap$, la de $x$:
\begin{flalign}
  x = \lambda \label{eq:xdelamda}
\end{flalign}
\begin{flalign}
  \textnormal{Por la ecuación } (\ref{eq:planopiprima}) && 2x+2y-z&=6       && \notag \\
  \textnormal{Despejamos } z                            && z&=2x+2y-6       && \notag \\
  \textnormal{Usando (\ref{eq:xdelamda})}               && z&=2\lambda+2y-6 && \label{eq:zdelamdaey}
\end{flalign}

\begin{flalign}
  \textnormal{Por la ecuación } (\ref{eq:planopi})     && 4x-y+3z&=2                           && \notag \\
  \textnormal{Usando (\ref{eq:xdelamda})}              && 4\lambda-y+3z&=2                     && \notag \\
  \textnormal{Usando (\ref{eq:zdelamdaey})}            && 4\lambda-y+3(2\lambda+2y-6)&=2       && \notag \\
  \textnormal{Distribuyendo}                           && 4\lambda-y+6\lambda+6y-18&=2         && \notag \\
  \textnormal{Agrupando y sumando}                     && 10\lambda+5y-18&=2                   && \notag \\
  \textnormal{Aislando } y                             && 5y&=2+18-10\lambda                   && \notag \\
  \textnormal{Agrupando y sumando}                     && 5y&=-10\lambda+20                    && \notag \\
  \textnormal{Dividiendo por 5}                        && y&=-2\lambda + 4 \label{eq:ydelamda} &&
\end{flalign}


Sólo queda $z$ para resolver
\begin{flalign}
  \textnormal{Por la ecuación (\ref{eq:zdelamdaey})}   && z &= 2\lambda+2y-6                   && \notag \\
  \textnormal{Usando (\ref{eq:ydelamda})}              && z &= 2\lambda+ 2(-2\lambda + 4)-6    && \notag \\
  \textnormal{Distribuyendo}                           && z &= 2\lambda-4\lambda + 8-6         && \notag \\
  \textnormal{Agrupando y sumando}                     && z &= -2\lambda+2 \label{eq:zdelamda} &&
\end{flalign}

Resumiendo: en las ecuaciones (\ref{eq:xdelamda}), (\ref{eq:ydelamda}) y (\ref{eq:zdelamda}) hemos hallado
la forma de obtener las coordenadas de los puntos de $\Lambda^\cap$ en función del parámetro $\lambda$:
\begin{flalign*}
  x &= \lambda && \\
  y &= -2\lambda + 4 && \\
  z &= -2\lambda+2 &&
\end{flalign*}
$
  \textnormal{Entonces } \Lambda^\cap = {(x,y,z),\lambda } \in \mathbb{R}^3  \text{ para los cuales se cumple: } \forall{\lambda}
  \begin{cases}
    x = \lambda       \\
    y = -2\lambda + 4 \\
    z = -2\lambda+2
  \end{cases}
$

Por fin llegamos a nuestra recta intersección $\Lambda^\cap$:
\begin{flalign}
  \Lambda^\cap =
  \{
  (x,y,z) \in  \mathbb{R}^3 \mid \forall \lambda \in \mathbb{R}: (
  x = \lambda,
  y = -2\lambda + 4,
  z = -2\lambda+2 )
  \}
\end{flalign}


\subsubsection{Forma vectorial de $\Lambda^\cap$ en función de $\lambda$}


Si tomamos
$\vec{x} = (x, y, z)$,
$\vec{u} = (0,4,2)$ y
$\vec{v} = (1,-2,-2)$
el conjunto de ecuaciones anterior se puede expresar como:
$$\vec{x} = \vec{u} + \lambda\vec{v}$$\\
O, en la extraña notación de los profesores de tu alumna ($\vec{x} = [\vec{v}] + \vec{u}$)`:
$$
  \vec{x} = [(1,-2,-2)] + (0,4,2)
$$


\subsubsection{Geogebra me da otra ecuación}

Si. Claro. Sería una casualidad que de la misma. Todo depende de una decisión personal: de donde sacamos el parámetro $\lambda$.
Lo que hizo Geogebra es tomar $z$ como parámetro, no $x$ como -tan inteligentemente- hiciste vos. 


La recta que obtiene Geogebra al intersecar los planos $\Pi$ y $\Pi'$ es $(1.67,0.67,-1.33) + \lambda(-5,10,10)$
que se puede expresar como $(\frac{5}{3},\frac{2}{3},-\frac{4}{3}) + \lambda(-5,10,10)$.

Analicemos esta situación. En primer lugar, como siempre, pongamos nombres a las cosas:
$$
\vec{x} = (x, y, z), 
\vec{u} = (0,4,2), 
\vec{v} = (1,-2,-2), 
\vec{w} = (\frac{5}{3},\frac{2}{3},-\frac{4}{3}),  
\vec{z} = (-5,10,10) \textnormal{ y } \lambda_1, \lambda_2 \in \mathbb{R}
$$
\begin{flalign}
  \textnormal{Nuestra recta } \Lambda^\cap \textnormal{ : }&& \vec{x} &= \vec{u} + \lambda_1\vec{v} \label{eq:nuestra}\\
  \textnormal{y la de Geogebra } \Lambda^G \textnormal{ : }&& \vec{x} &= \vec{w} + \lambda_2\vec{z} \label{eq:geogebra}&&
\end{flalign}

Pero ambas ecuaciones (\ref{eq:nuestra}) y (\ref{eq:geogebra}) aluden a la misma recta. Es decir que
$\Lambda^\cap = \Lambda^G$.

¿Porqué? Porque se cumplen dos condiciones:

La primera condición es que \textbf{cada vector director es combinación lineal del otro}. 
Si hacemos las cuentas veremos que $\vec{v} = -\frac{1}{5}\vec{z}$.
Es decir que ambos están ``apuntando a la misma dirección'' (Aunque con distinto sentido, pero esto es irelevante, sólo un poco molesto). 

La segunda condición es que \textbf{ambos puntos de paso $\vec{u}$ y $\vec{w}$ están en la misma recta}.
Si restamos $\vec{u}$ a $\vec{w}$ obtenemos $\vec{d}=(\frac{5}{3},-\frac{10}{3},-\frac{10}{3})$ que es el vector director de la recta
que ambos definen. 
A simple vista encontramos que:
\begin{flalign*}
  \vec{d} \textnormal{ es una combinacion lineal de } \vec{v}           && \vec{d} &= \frac{5}{3}\vec{v} &&\\
  \vec{d} \textnormal{ es tambien es una combinacion lineal de } \vec{z} && \vec{d} &= -\frac{1}{3}\vec{z} &&
\end{flalign*}

Es decir que \textbf{la diferencia entre los puntos de paso se puede expresar como una combinación lineal } de cualquiera de los directores.
 
Este resultado era una condición obligatoria para llegar a la conclusión de que ambos puntos están en $\Lambda^\cap$ y también en $\Lambda^G$.
Y así poder afirmar que $\Lambda^\cap = \Lambda^G$.

Gráficamente lo podemos ver en https://www.geogebra.org/3d/a3scf54u.


\subsubsection{Verificación de que $\Lambda^\cap$ es realmente la intersección entre $\Pi$ y $\Pi'$}
Sabemos que para cada $x$, para cada $y$ y para cada $z$ en $\Lambda^\cap$ también se
cumplen las condiciones impuestas por los planos $\Pi$ $\rightarrow$  $4x-y+3z=2$ y $\Pi'$ $\rightarrow$ $2x+2y-z=6$.
Esto se debe, recordemos, al hecho de que, al ser $\Lambda^\cap$ la intersección entre
dichos planos sus puntos están en ambos.
Cualquiera que sea $\lambda$ se deben cumplir, entonces, todas las ecuaciones involucradas
en las definiciones de $\Lambda^\cap$, $\Pi$ y $\Pi'$.

Comencemos verificando el plano $\Pi$ $\rightarrow$  $4x-y+3z=2$. Usando la definición hallada de $\Lambda^\cap$ podemos
reemplazar las coordenadas por sus correspondientes funciones de $\lambda$:
\begin{flalign*}
  4x-y+3z&=2 \\
  4(\lambda)-(-2\lambda + 4)+3(-2\lambda+2)&=2 \\
  4\lambda+2\lambda-6\lambda+6-4&=2 \\
  2&=2
\end{flalign*}

El reemplazo de las coordenadas en el plano $\Pi$ $\rightarrow$  $4x-y+3z=2$ por las funciones de $\Lambda^\cap$ nos llevó por buen
camino. Ahora podemos afirmar, sin temor a equivocarnos, que $2=2$. ¿Podemos decir lo mismo
en el caso del plano $\Pi'$ $\rightarrow$ $2x+2y-z=6$? Veamos:
\begin{flalign*}
  2x+2y-z&=6 \\
  2(\lambda)+2(-2\lambda+4)-(-2\lambda+2)&=6 \\
  2\lambda-4\lambda+8+2\lambda-2&=6 \\
  2\lambda-4\lambda+2\lambda+8-2&=6 \\
  6&=6
\end{flalign*}

¡Perfecto! Matamos tres pájaros de un tiro: No solo verificamos que $\Lambda^\cap = \Pi \cap \Pi'$ sino que
además aprendimos que $2=2$ y $6=6$.


\subsubsection{Conclusión de la búsqueda de la intersección $\Lambda^\cap$}


$$\Lambda^\cap =
  \{
  (x,y,z) \in  \mathbb{R}^3 \mid \forall \lambda \in \mathbb{R}: (
  x = \lambda,
  y = -2\lambda + 4,
  z = -2\lambda+2 )
  \}
$$


\subsection{Plano tangente $\Pi^\perp$ a la intersección $\Lambda^\cap$}


\subsubsection{Obtención de la ecuación que define a $\Pi^\perp$}


Podemos expresar nuestra recién encontrada $\Lambda^\cap$ de la siguiente manera,
si tomamos
$\vec{x} = (x, y, z)$,
$\vec{u} = (0,4,2)$ y
$\vec{v} = (1,-2,-2)$:
$$\vec{x} = \vec{u} + \lambda\vec{v}$$\\
O, en la extraña notación de los profesores de tu alumna ($\vec{x} = [\vec{v}] + \vec{u}$)`:
$$
  \vec{x} = [(1,-2,-2)] + (0,4,2)
$$
Prestemos atención al hecho de que $\vec{v} = (1,-2,-2)$
es el vector \textbf{director} de $\Lambda^\cap$ y el vector \textbf{normal} a
infinitos planos que llenan el espacio $\mathbb{R}^3$.
Esos planos tienen en común una ecuación que vamos a encontrar a continuación.

Recordemos que la ecuacion general del plano es $ax+by+cz=d$. De ella podemos extraer el vector $\vec{c} = (a,b,c)$. Y entonces podemos
expresar la ecuación general del plano como $\langle(a,b,c),(x,y,z)\rangle = d$. Si llamamos $\vec{x}$ a $(x,y,z)$ podemos expresar dicha ecuación
como $\langle\vec{x},\vec{c}\rangle = d$. En el origen, que es el lugar en donde $d=0$, queda claro por qué el vector $\vec{c}$ de los coeficientes se
llama vector \textbf{normal} al plano.

Si buscamos un plano ortogonal a nuestra $\Lambda^\cap$, entonces, ya conocemos sus coeficientes $(a,b,c)$ porque son los los valores del
vector director de $\Lambda^\cap$ que son  $(1,-2,-2)$. Es decir $a=1$, $b=-2$ y $c=-2$.

Podemos reemplazarlos entonces en la ecuacion general para ir avanzando en nuestra búsqueda:

\begin{flalign*}
  ax+by+cz&=d \\
  (1)x+(-2)y+(-2)z&=d \\
  x-2y-2z&=d \\
\end{flalign*}

Ok. Los infinitos planos normales a $\Lambda^\cap$ se especifican con la ecuación que tienen en común ($x-2y-2z=d$).
Recordemos que $x$, $y$ y $z$ son las variables independientes, pueden tener cualquier valor en $\mathbb{R}^3$,
aunque solo aquellos valores que cumplen $x-2y-2z=d$ hacen que el punto que definen
esté en el plano determinado por un cierto valor de $d$.

¿Cual es, entonces, el valor de $d$ que determina el plano $\Pi^\perp$ buscado?

Afortunadamente, por la definición del problema ("...y que pase por $(4,2,-1)$"),
tenemos un punto que sabemos que está en $\Pi^\perp$. Vamos a usarlo entonces para que queden determinados $x$, $y$ y $z$ en la ecuación de
los planos ortogonales a $\Lambda^\cap$ y así quede aislada la única incógnita, $d$.

\begin{flalign*}
  x-2y-2z&=d \\
  d&=x-2y-2z \\
  d&=(4)-2(2)-2(-1) \\
  d&=4-4+2 \\
  d&=2
\end{flalign*}

Conocemos ahora la única incógita que nos quedaba ($d$), sabemos ahora que su valor es $2$, por lo tanto nuestro plano $\Pi^\perp$ es:
\begin{flalign}
  \Pi^\perp = \{ (x,y,z) \in \mathbb{R}^3 \mid x-2y-2z=2\} \label{eq:resultado}
\end{flalign}

\subsubsection{Verificación de que $\Pi^\perp \perp \Lambda^\cap$}
Es directa y bastante obvia, el vector director de $\Lambda^\cap$
es $\vec{v} = (1,-2,-2)$ que coincide con el vector normal de $\Pi^\perp$ $\rightarrow$  $x-2y-2z=-2$
que es $(1,-2,-2)$.


\subsubsection{Verificación de que $(4,2,-1) \in \Pi^\perp$}

\begin{flalign*}
  x-2y-2z&=2 \\
  (4)-2(2)-2(-1)&=2 \\
  4-4+2&=2 \\
  2&=2
\end{flalign*}

Eso ya lo sabíamos.

\subsubsection{¿Es (4,2,-1) la intersección de $\Pi^\perp$ y  $\Lambda^\cap$?}
Veamos:
\begin{flalign*}
  x &= \lambda && \\
  y &= -2\lambda + 4 && \\
  z &= -2\lambda+2 \\
  x-2y-2z&=2
\end{flalign*}
se tienen que cumplir todas en  $\Pi^\perp \cap \Lambda^\cap$

Aislamos y encontramos el valor de $\lambda$ reemplazando, en la ecuación del plano $\Pi^\perp$, las coordenadas por sus correspondientes funciones de $\lambda$:
\begin{flalign*}
  x-2y-2z&=2 \\
  \lambda-2(-2\lambda + 4)-2(-2\lambda+2)&=2 \\
  \lambda+4\lambda - 8+4\lambda-4&=2 \\
  (1+4+4)\lambda &= 2+8+4 \\
  9\lambda &= 14 \\
  \lambda &= \frac{14}{9}
\end{flalign*}

Ahora que sabemos $\lambda$ de $\Pi^\perp \cap \Lambda^\cap$ podemos obtener sus coordenadas.

\begin{flalign*}
  && x = \lambda \\
  && x = \frac{14}{9} && x = \frac{14}{9} \\
  && y = -2\lambda + 4 \\
  && y = -2\frac{14}{9} + 4 && y = \frac{8}{9}\\
  && z = -2\lambda+2 \\
  && z = -2\frac{14}{9} + 2 && z = -\frac{10}{9}\\
\end{flalign*}

Llamemos $P_i$ a la intersección $\Pi^\perp \cap \Lambda^\cap$. Acabamos de obtener:
\begin{flalign}
  P_i = (\frac{14}{9},\frac{8}{9},-\frac{10}{9})
\end{flalign}

¿Está bien hecha la cuenta? Veamos si $P_i \in \Pi^\perp$

\begin{flalign*}
  x-2y-2z&=2 \\
  (\frac{14}{9})-2(\frac{8}{9})-2(-\frac{10}{9})&=2 \\
  \frac{14}{9}-\frac{16}{9}+\frac{20}{9}&=2 \\
  \frac{18}{9}&=2 \\
  2&=2
\end{flalign*}

Efectivamente $P_i \in \Pi^\perp$. Por lo tanto la respuesta a la pregunta ¿(4,2,-1)=$\Pi^\perp\cap\Lambda^\cap$? es \textbf{no}, $(4,2,-1)$ no es la intersección
de $\Pi^\perp$ y  $\Lambda^\cap$, ese honor le corresponde a $(\frac{14}{9},\frac{8}{9},-\frac{10}{9})$.


\section{Respuesta}
Siendo $\Pi$ $\rightarrow$ $4x-y+3z=2$ y $\Pi' \rightarrow 2x+2y-z=6$
y $\Lambda^\cap = \Pi \cap \Pi'$.




Buscábamos
$\Pi^\perp \mid \Pi^\perp \perp \Lambda^\cap, (4,2,-1) \in \Pi^\perp$.

Lo encontramos. Es: $\Pi^\perp = \{ (x,y,z) \in \mathbb{R}^3 \mid x-2y-2z=2\}$. Está en la ecuación (\ref{eq:resultado}).

Misión cumplida.

\newpage

\tableofcontents

\end{document}

